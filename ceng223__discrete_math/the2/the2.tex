\documentclass[11pt]{article}
\usepackage[utf8]{inputenc}
\usepackage{float}
\usepackage{amsmath}


\usepackage[hmargin=3cm,vmargin=6.0cm]{geometry}
%\topmargin=0cm
\topmargin=-2cm
\addtolength{\textheight}{6.5cm}
\addtolength{\textwidth}{2.0cm}
%\setlength{\leftmargin}{-5cm}
\setlength{\oddsidemargin}{0.0cm}
\setlength{\evensidemargin}{0.0cm}

\DeclareUnicodeCharacter{2212}{-}
\begin{document}

\section*{Student Information } 
%Write your full name and id number between the colon and newline
%Put one empty space character after colon and before newline
Author : mergen\\


% Write your answers below the section tags
\section*{Answer 1}
\textbf{a)}\\
(i)\hspace{0.15cm} D = A $\cap$ ( B $\cup$ C )\\
(ii)\hspace{0.05cm} E = (A $\cap$ B ) $\cup$ C\\
(iii) F = ( A - B ) $\cup$ (A $\cap$ C )\\ 
\\
\textbf{b)}\\
\textbf{Let A = \{e,n,t\}, B = \{t,e,r\}, and C = \{s,e,r\}} \\
\\
\\
(i) ( A X B ) X C = \{(e,t),(e,e),(e,r),(n,t),(n,e),(n,r),(t,t),(t,e),(t,r)\} X \{s,e,r\} = \{(e,t,s),(e,e,s),(e,r,s),(n,t,s),\\ 
\hspace{1cm}(n,e,s),(n,r,s),(t,t,s),(t,e,s),(t,r,s),(e,t,e),(e,e,e),(e,r,e),(n,t,e),(n,e,e),(n,r,e),(t,t,e),(t,e,e),(t,r,e),(e,t,r),(e,e,r),\\
(e,r,r),(n,t,r),(n,e,r),(n,r,r),(t,t,r),(t,e,r),(t,r,r)\} = \{e,n,t\} X \{(t,s),(t,e),(t,r),(e,s),(e,e),(e,r),(r,s),(r,e),(r,s)\} = A X ( B X C )\\
\\
(ii) (A $\cap$ B ) $\cap$ C = \{e,t\}$\cap$ \{s,e,r\} = \{e\} = \{e,n,t\}$\cap$\{e,r\} = A $\cap$ ( B $\cap$ C ) \\
\\
(iii) ( A$\oplus$ B ) $\oplus$ C =\{n,r\}$\oplus$\{s,e,r\} =\{s,e,n\} = \{e,n,t\}$\oplus$ \{t,s\} = \{e,n,s\} = A $\oplus$ ( B $\oplus$ C ) \\

\section*{Answer 2}
\textbf{a)}\\
Take any x $\in$ A$_\text{0}$, then we have \textbf{\textit{f}}(x) $\in$ \textbf{\textit{f}}(A$_\text{0}$). By definition, \textbf{\textit{f$^\text{-1}$}}\textbf{\textit{(f}}(A$_\text{0}$)\textbf{)} = \{a $\mid$ \textbf{\textit{f}}(a) $\in$ \textbf{\textit{f}}(A$_\text{0}$) \}. Since \textbf{\textit{f}}(x) $\in$ \textbf{\textit{f}}(A$_\text{0}$), x satisfies the condition(1):  \textbf{\textit{f}}(a) $\in$ \textbf{\textit{f}}(A$_\text{0}$); we have x $\in$ \textbf{\textit{f$^\text{-1}$}}\textbf{\textit{(f}}(A$_\text{0}$)\textbf{)}. We have proved x $\in$ A$_\text{0}$ $\xrightarrow{}$ x $\in$ \textbf{\textit{f$^\text{-1}$}}\textbf{\textit{(f}}(A$_\text{0}$)\textbf{)}. That means A$_\text{0}$ $\subseteq$ \textbf{\textit{f$^\text{-1}$}}\textbf{\textit{(f}}(A$_\text{0}$)\textbf{)}. Now, suppose \textbf{\textit{f}} injective. We want to prove that A$_\text{0}$ = \textbf{\textit{f$^\text{-1}$}}\textbf{\textit{(f}}(A$_\text{0}$)\textbf{)}. Suppose not; that means we can find an x $\in$ \textbf{\textit{f$^\text{-1}$}}\textbf{\textit{(f}}(A$_\text{0}$)\textbf{)} such that x $\notin$ A$_\text{0}$. Because we already know A$_\text{0}$ $\subseteq$ \textbf{\textit{f$^\text{-1}$}}\textbf{\textit{(f}}(A$_\text{0}$)),  x $\in$ \textbf{\textit{f$^\text{-1}$}}\textbf{\textit{(f}}(A$_\text{0}$)\textbf{)} $\xrightarrow{}$ x satisfies the contidion(1) then \textbf{\textit{f}}(x) = u $\in$ \textbf{\textit{f}}(A$_\text{0}$). u $\in$ \textbf{\textit{f}}(A$_\text{0}$) means, there exists a y $\in$ A$_\text{0}$ with \textbf{\textit{f}}(y) = u. Since \textbf{\textit{f}} is injective and \textbf{\textit{f}}(x) = u = \textbf{\textit{f}}(y), we must have x=y. But x $\notin$ A$_\text{0}$ and y $\in$ A$_\text{0}$; contradiction. \\
\\
\textbf{b)}\\
Take any y $\in$ \textbf{\textit{f$^\text{-1}$}}\textbf{\textit{(f}}(B$_\text{0}$)\textbf{)} $\xrightarrow{}$ y = \textbf{\textit{f}}(x) for some x $\in$ \textbf{\textit{f$^\text{-1}$}}(B$_\text{0}$). x $\in$ \textbf{\textit{f$^\text{-1}$}}(B$_\text{0}$) = \{a $\mid$ \textbf{\textit{f}}(a) $\in$ B$_\text{0}$\} means x satisfies the condition(1); \textbf{\textit{f}}(x) $\in$ B$_\text{0}$ $\xrightarrow{}$ y $\in$ B$_\text{0}$. So, \textbf{\textit{f(}}\textbf{\textit{f$^\text{-1}$}}(B$_\text{0}$)\textbf{)} $\subseteq$ B$_\text{0}$. Suppose \textbf{\textit{f}} is surjective, but \textbf{\textit{f(}}\textbf{\textit{f$^\text{-1}$}}(B$_\text{0}$)\textbf{)} $\neq$ B$_\text{0}$. That means we can find a y $\in$ B$_\text{0}$ such that y $\notin$ \textbf{\textit{f(}}\textbf{\textit{f$^\text{-1}$}}(B$_\text{0}$)\textbf{)}. By surjectivity, there exists an x such that \textbf{\textit{f}}(x) = y $\xrightarrow{}$ x satisfies the condition(1); \textbf{\textit{f}}(x) $\in$ B$_\text{0}$ $\xrightarrow{}$ x $\in$ \textbf{\textit{f$^\text{-1}$}}(B$_\text{0}$) $\xrightarrow{}$  \textbf{\textit{f}}(x) $\in$ \textbf{\textit{f(}}\textbf{\textit{f$^\text{-1}$}}(B$_\text{0}$)\textbf{)} $\xrightarrow{}$ y $\in$ \textbf{\textit{f(}}\textbf{\textit{f$^\text{-1}$}}(B$_\text{0}$)\textbf{)}; hence contradiction. \\


\section*{Answer 3}
Let A be a non-empty set.\\
(i) A is countable \\
(ii)  There is a surjective function \textbf{\textit{f}} : Z$^\text{+}$ $\xrightarrow{}$ A  \\
(iii) There is an injective function \textbf{\textit{g}} : A $\xrightarrow{}$ Z$^\text{+}$ \\
\\
(i) $\xrightarrow{}$ (ii): If A is countably infinite, then there exists a bijection \textbf{\textit{f}} : Z$^\text{+}$ $\xrightarrow{}$ A and then (ii) follows. If A is finite, then there is bijection \textit{h} : \{1,2,...,n\} $\xrightarrow{}$ A for some n. Then the function \textbf{\textit{f}} : Z$^\text{+}$ $\xrightarrow{}$ A defined by;\\
$\displaystyle f(t) = \begin{cases}
\textit{h}(t) & \text{1 $\leq$ t $\leq$ n,} \\ 
\textit{h}(n) & \text{$t > n$} \\ 
\end{cases}$
\hspace{0.4cm}is a surjection.\\
\vspace{0.6cm} \\
(ii) $\xrightarrow{}$ (iii): Assume that \textbf{\textit{f}} : Z$^\text{+}$ $\xrightarrow{}$ A is a surjection. We claim that there is an injection \textbf{\textit{g}} : A $\xrightarrow{}$ Z$^\text{+}$. To define \textbf{\textit{g}} note that if a $\in$ A, then \textbf{\textit{f}}$^\text{-1}$(\{a\}) $\neq$ $\emptyset$. Hence we set \textbf{\textit{g}}(a) = $\textit{min}$\textbf{\textit{f}}$^\text{-1}$(\{a\}). Now note that if a $\neq$ a$\ensuremath{'}$, then the sets \textbf{\textit{f}}$^\text{-1}$(\{a\}) $\cap$ \textbf{\textit{f}}$^\text{-1}$(\{a$\ensuremath{'}$\}) = $\emptyset$ which implies min$^\text{-1}$(\{a\}) $\neq$ min$^\text{-1}$(\{a$\ensuremath{'}$\}). Hence \textbf{\textit{g}}$^\text{-1}$(a) $\neq$ \textbf{\textit{g}}$^\text{-1}$(a$\ensuremath{'}$) and \textbf{\textit{g}} : A $\xrightarrow{}$ Z$^\text{+}$ is an injective function.\\
\vspace{0.6cm} \\
(iii) $\xrightarrow{}$ (i): Assume that \textbf{\textit{g}} : A $\xrightarrow{}$ Z$^\text{+}$ is an injection. We want to show that A is countable. Since  \textbf{\textit{g}} : A $\xrightarrow{}$ g(A) is a bijection and g(A) $\subset$ Z$^\text{+}$, proposition "Any subset of a countable set is countable." implies that A is countable.\\
\section*{Answer 4}
\textbf{a)}\\
Let \textbf{F} be the set containing finite binary strings and $\epsilon$ denote the empty sequence(the sequence with no terms). Then, the sequence \textbf{F} = \{$\epsilon$,0,1,00,01,10,11,000,001,010,011,...\} in which the binary sequences of length 0 are listed, then the binary sequences of length 1 are listed in increasing numeric order, then the binary sequences of length 2 are listed in increasing numeric order, and so on, contains every finite binary sequence exactly once.\\
\\
\textbf{b)}\\
Let \textbf{I} be the set of all infinite sequences of 0s and 1s. We use Cantor’s diagonal argument. So we assume (toward
a contradiction) that we have an enumeration of the elements of \textbf{I}, say as \textbf{I} = \{i$_\text{1}$i$_\text{2}$i$_\text{3}$...\} where each s$_\text{n}$ is an infinite sequence of 0s and 1s. We will write s$_\text{1}$ = s$_\text{1,1}$s$_\text{1,2}$s$_\text{1,3}$ · · · , s$_\text{2}$ = s$_\text{2,1}$s$_\text{2,2}$s$_\text{2,3}$ · · · , and so on; so s$_\text{n}$ = s$_\text{n,1}$s$_\text{n,2}$s$_\text{n,3}$ · · · . So we denote the \textbf{\textit{m}}th element of s$_\text{n}$ by s$_\text{n,m}$. Now we create a new sequence t = t$_\text{1}$t$_\text{2}$t$_\text{3}$t$_\text{4}$ · · · of 0s and 1s as follows: t$_\text{n}$ = s$_\text{n,n}$ − 1 (so t$_\text{n}$ = 1 if s$_\text{n,n}$ = 0 and t$_\text{n}$ = 0 if the s$_\text{n,n}$ is 1).  It is clear that t is an element of \textbf{I} - it is an infinite sequence of 0s and 1s. However, we will now see that t is not in the list above. Suppose that t = i$_\text{k}$ for some value of k. Then t$_\text{k}$ = t$_\text{k,k}$, but by the construction, t$_\text{k}$ $\neq$ i$_\text{k,k}$, so this is not possible. We
conclude that \textbf{I} is not countable, is uncountable.\\
\section*{Answer 5}
\textbf{a)}\\
f(n) is $\Theta$(g(n)), where f(n)=nlog(n), g(n)=log(n!) and there exist constants c and k, if:\\
$\hspace*{1cm}$(i) f(n) is O(g(n)) $\xrightarrow{}$ f(n) $\leq$ cg(n) whenever $n > k$\\
$\hspace*{3cm}$ log(n!) = log(1)+log(2)+...+log(n)\\
$\hspace*{4.3cm}$ $\leq$ log(n)+log(n)+...+log(n)\\
$\hspace*{3cm}$ log(n!) $\leq$ nlog(n); where c = 1 and $n > 1$\\
\\
$\hspace*{1cm}$ (ii) f(n) is $\Omega$(g(n)) $\xrightarrow{}$ f(n) $\geq$ cg(n)\\
$\hspace*{3cm}$ log(n!) = $\sum_{i=1}^{n} log(i)$ \\
$\hspace*{4.3cm}$ $\geq$ $\sum_{i=n/2}^{n} log(i)$ \\
$\hspace*{4.3cm}$ $\geq$ $\sum_{i=n/2}^{n} log(n/2)$ \\
$\hspace*{4.3cm}$ $\geq$ $\frac{n}{2}$log(n/2)\\
$\hspace*{4.3cm}$ = $\frac{n}{2}$(log(n) - log(2)); where c = $\frac{1}{2}$ and $n > 1$\\
$\hspace*{4.3cm}$ = $\Omega$ (nlog(n))\\
\textbf{b)}\\
\[ 2^n = \underbrace{2*2*...*2*2}_\text{n times} \]
\[ n! = \underbrace{1*2*...*(n-1)*n}_\text{n elements} \]
\hspace*{4cm}$\frac{2^\text{n}}{n!}$ = $\frac{2}{1}$*$\frac{2}{2}$*$\frac{2}{3}$*...*$\frac{2}{n-1}$*$\frac{2}{n}$\\
\\
\hspace*{4cm}$\frac{2^\text{n}}{n!}$ $\leq$ $\frac{2}{1}$*$\frac{2}{2}$*1*...*1*$\frac{2}{n}$ = $\frac{4}{n}$ \\
\\
\hspace*{1cm}Hence 2$^\text{n}$ is O(n!) such that n! grows faster than 2$^\text{n}$ ($\mid$2$^\text{n}$$\mid$$\leq$ c$\mid$n!$\mid$); where c = 1 and n $\geq$ 4\\
\end{document}
